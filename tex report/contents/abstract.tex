\newpage
\thispagestyle{plain}
\begin{center}
    \LARGE
    \textbf{Abstract}
    \vspace{0.8cm}
\end{center}
Farm water reservoirs support agriculture by storing water during high rainfall or streamflow periods for use during low periods, helping to smooth out supply-demand imbalances. Many small reservoirs in the UK are unregistered as they do not require a permit below 25\,000\,m$^3$ water capacity. It is important to have a register of these reservoirs to ensure responsible water management and to mitigate water scarcity risks. Traditional monitoring methods, such as manual surveys, are time-consuming and labour-intensive, particularly in remote areas. This project develops a machine learning software capable of classifying small water reservoirs in east England from satellite imagery.

First, the Navigable Automated Labelling Interface for Regions of Attention (NALIRA) was developed as an automated approach to generating labelled training data for machine learning models. In NALIRA, satellite images are automatically manipulated and cloud masked, then several water detection indices are calculated and plotted. The user is then prompted to label water reservoirs and water bodies in small “chunks” of the image, after which the selections are segmented and saved. NALIRA transforms the traditionally laborious task of manual image labelling into a streamlined workflow by automating data preparation, significantly accelerating training data generation. The Keras Reservoir Identification Sequential Platform (KRISP) was trained to classify small water reservoirs with the data from NALIRA.

While the development of NALIRA was more time-intensive than expected, the final software is highly modular and scalable, featuring a navigable GUI for the user to easily label large quantities of data. In total, NN small water reservoirs were found in East England, estimated with a confidence of MM\% for the period of March 2025. This area and time period were chosen for their low cloud cover and high concentration of farmland. KRISP successfully achieved XX\% accuracy, YY\% precision, and ZZ\% recall in classifying small water reservoirs, and AA\% accuracy, BB\% precision, and CC\% recall in classifying non-reservoir water bodies. 

NALIRA can be adapted to generate labelled training data for any image-based application in supervised machine learning, expediting the ground-up development of small machine vision models in engineering, medicine, and computer science. KRISP is a tool that has generated the first and only estimate of the number of small reservoirs in East England, and can therefore be used to support water management and policy decisions. 

Future work includes using more sophisticated water and cloud detection techniques, increasing quantity of training data, and a complete transition to Google Earth Engine as a data source, which would allow the program to access petabytes of web-based satellite data.
