

\section{Overview}
The overarching method employed is fundamentally a four-step process which aligns with the objectives defined in the introductory section. 

\subsection{NALIRA: Objectives 1 \& 2}
\subsubsection{}
\subsubsection{Implication on Aim and Objectives}

\subsection{KRISP Trainer: Objective 3}
\subsubsection{Implication on Aim and Objectives}

\subsection{KRISP: Objective 4}
\subsubsection{Implication on Aim and Objectives}


\section{Contextualisation Within the Field}


\section{Limitations}
\subsection{Cloud Masking Approach}
\subsubsection{Analysis of Impact on Results}

\subsection{Limited Training Data}
\subsubsection{Analysis of Impact on Results}

\subsection{Computational Power}
\begin{itemize}
    \item Satellite imagery is composed of several high-resolution images of a specific wavelength of light. 
    \item This means that each single "image", which in reality is packaged as a folder of images, contains many hundreds of megabytes (MBs), and sometimes over a gigabyte (GB) worth of data. 
    \item Handling this large quantity of data can be highly computationally intensive and slow if implemented incorrectly, therefore, in the explanation of the programs developed for this study, it should be noted that a majority of the time spent working on the code was to optimise and refine memory usages, Input/Output (IO) handlings, and various other efficiency improvements. 
\end{itemize}
\subsubsection{Analysis of Impact on Results}
Low

