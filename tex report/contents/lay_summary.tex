\newpage
\thispagestyle{plain}
\begin{center}
    \LARGE
    \textbf{Lay Summary}
    \vspace{0.8cm}
\end{center}

Many farms use small reservoirs to store water, which is vital for growing crops, especially during dry periods. However, countless small reservoirs across the UK aren't officially registered because they fall below a certain size limit. This lack of knowledge makes it hard to manage water resources responsibly and understand the full impact of water use in agriculture. Finding all these small reservoirs manually across the country would be incredibly time-consuming and expensive.

This project tackles this problem by using satellite images and machine learning (ML). I developed new computer software tools to automatically process satellite pictures of East England, a major farming area. One tool, named NALIRA, helps prepare the images and makes it much faster for users to identify examples of reservoirs. Another program, named KRISP, trained using these examples, then learns to automatically spot and count small reservoirs across the region.

The result is the first estimate of small reservoir numbers in East England, providing valuable information for better water management planning and environmental monitoring. The image preparation tool developed could also speed up similar ML projects in other fields that rely on analysing images.